\documentclass{article}
\usepackage{fancyhdr}
\usepackage{lastpage}
\usepackage{listings}
\usepackage[margin=1in]{geometry}
\usepackage{xcolor}

\lstset{
	language=C,
	basicstyle=\footnotesize\ttfamily,
	commentstyle=\footnotesize\ttfamily \color{black!60},
	numbers=left,
	numberstyle=\tiny,
	stepnumber=1,
	numbersep=5pt,
	frame=lines,
	tabsize=4,
	breaklines=true,
	breakatwhitespace=true,
	showspaces=false,
	captionpos=b
}

\pagestyle{fancy}
\lhead{ECE-C353}
\chead{Homework 4 Histogram Improvements}
\rhead{Sean Barag}
\cfoot{\thepage\ of \pageref{LastPage}}

\newcommand{\lst}[2]{
	\begin{center}
	\parbox{.6\textwidth}{
	\lstinputlisting[firstline=#1, lastline=#2, firstnumber=#1]{../histogram.c}}
	\end{center}
}

\newcommand{\ttt}[1]{\texttt{#1}}

\title{ECE-C353: Systems Programming \\ Homework 4 Histogram Improvements}
\author{Sean Barag \\ \ttt{<sjb89@drexel.edu>}}
\date{\today}

\begin{document}
\maketitle
\section{Design}
My multi-threaded implementation of histogram creation was based on code that
was previously submitted to compute the sum of~5,000,000 floats.  By splitting
the input data into several groups, a divide-and-conquer approach can be
applied.

\subsection{Custom Type}
My solution required the creation of a custom object type, through which I passed each thread's arguments.
%
\lst{21}{31}
%
Each thread received several pieces of information, each of which was represented by an object within the \ttt{THREAD\_ARGS} type:

\begin{center}
\parbox{.85\textwidth}{
\begin{description}
	\item[threadID]{Integer identifier for the thread.  The first thread
	spawned will receive ID~0, and subsequent threads will
	receive~1,~2, $\ldots$ , \ttt{num\_threads} - 1.}
	\item[start]{Index within the input data at which the thread will begin to
	process data.}
	\item[count]{Number of elements the thread shall process.}
	\item[src]{Pointer to an array of integers that will be the source of data
	to be processed.}
	\item[dst]{Pointer to an array of integers that will serve as the end
	destination for the processed data.}
	\item[buf]{Pointer to an array of integers that will be used to buffer the
	thread's processed data until its completion.}
\end{description}
}
\end{center}

\subsection{Initialization}
The multi-threaded implementation has an entrance point at the provided call to
\ttt{compute\_using\_pthreads}, now at line~94.  It begins by simply
initializing a series of variables and a semaphore for the implementation's shared histogram:
%
\lst{134}{145}
%
It then allocates \ttt{num\_threads} a two-dimensional array consisting of
\ttt{num\_threads} single-dimensional arrays, each with
\ttt{histogram\_size} elements.  Each one-dimensional array will act as a
buffer for a thread's processed values, allowing each thread to make changes to
the shared histogram only once.  This reduces contention and increases
performance.  The elements of each buffer array are initialized to zero, and
the process of spawning threads begins.
%
\lst{147}{155}

\subsection{Thread Creation}
In preparation for the creation of a thread, a new instance of
\ttt{THREAD\_ARGS} is allocated and filled with the appropriate values.  Of
note in this process is line~168, which handles the case that the number of
elements to bin is not an integer multiple of the number of threads being
spawned.  The line simply checks if the thread currently being built is the
last one.  If it is not, the number of elements to bin is kept at its default
value; if it \emph{is} the last thread, the modulus of the number of threads
and number of elements is added to the default.  I.E. the last thread will bin
whatever didn't fit cleanly into the other threads.  This adds at most
\ttt{num\_threads} - 1 elements to the final thread, which is negligible for all
cases examined here.
%
\lst{157}{185}
%
Since the source and destination pointers do not vary between threads,
performance could be increased marginally by statically defining these values,
but they were left this way for the sake of clarity.  Once each thread is
successfully spawned, \ttt{compute\_using\_pthreads} waits for \emph{all} of
the worker threads to return from their operation of the \ttt{do\_histogram}
method.  At this point, it returns to \ttt{main()}.

\subsection{Processing}
The process that each thread goes through to bin data elements is actually
quite simple.  The thread begins by casting the arguments to a usable form, and
then immediately bins its assigned range of input values to its buffer array,
using the same logic as the provided reference method.
%
\lst{187}{206}
%
When a thread has binned its assigned data, it either locks the master
histogram or waits for another thread to release it via \ttt{sem\_wait}.  A
thread's buffered histogram is merged into the master histogram once the lock
is obtained, and the lock is immediately released when it is no longer needed.
Finally, each thread exits when it has successfully released~(via
\ttt{sem\_post}) the lock on the shared histogram.

\end{document}

