\documentclass{article}
\usepackage{fancyhdr}
\usepackage{lastpage}
\usepackage{listings}
\usepackage[margin=1in]{geometry}
\usepackage{xcolor}
\usepackage{gnuplot-lua-tikz}
\usepackage{subfig}
\usepackage{siunitx}
\usepackage{float}
\usepackage{siunitx}

\lstset{
	language=C,
	basicstyle=\footnotesize\ttfamily,
	commentstyle=\footnotesize\ttfamily \color{black!60},
	numbers=left,
	numberstyle=\tiny,
	stepnumber=1,
	numbersep=5pt,
	frame=leftline,
	tabsize=4,
	breaklines=true,
	breakatwhitespace=true,
	showspaces=false,
	captionpos=b
}

\pagestyle{fancy}
\lhead{ECE-C353}
\chead{Final Project}
\rhead{Sean Barag}
\cfoot{\thepage\ of \pageref{LastPage}}

\newcommand{\clst}[2]{
	\begin{center}
	\parbox{.6\textwidth}{
	\lstinputlisting[firstline=#1, lastline=#2, firstnumber=#1]{../src/work_crew.c}}
	\end{center}
}

\newcommand{\hlst}[2]{
	\begin{center}
	\parbox{.6\textwidth}{
	\lstinputlisting[firstline=#1, lastline=#2, firstnumber=#1]{../src/work_crew.h}}
	\end{center}
}

\newcommand{\chompclst}[2]{
	\begin{center}
	\parbox{.6\textwidth}{
	\lstinputlisting[firstline=#1, lastline=#2, firstnumber=#1]{chomped.c}}
	\end{center}
}

% I NEED SUBSUBSUBSECTIONS (``paragraphs''), DAMMIT
\setcounter{secnumdepth}{5}
% Force TOC to descend 5 levels
\setcounter{tocdepth}{5}

\newcommand{\ttt}[1]{\texttt{#1}}
\newcommand{\tbf}[1]{\textbf{#1}}

\title{ECE-C353: Systems Programming \\ Final Project}
\author{Sean Barag \\ \ttt{<sjb89@drexel.edu>}}
\date{\today}

\begin{document}
\maketitle
\tableofcontents
\newpage
\section{Design}
The design for this multi-threaded recursive text search was based mostly on
provided code, but included heavy modification to prevent deadlocks.

\subsection{Header File}
All global instance, function, and type definitions were placed in the header
file~\ttt{work\_crew.h}, to improve the readability of the main source
file~\ttt{work\_crew.c}.
%
\hlst{1}{10}
%
These objects were declared globally, with the following intended meanings:
%
\begin{center}
\parbox{.85\textwidth}{
\begin{description}
	\item[MAX\_THREADS]{Maximum number of threads to possibly create.}
	\item[num\_threads]{Number of threads that have been spawned at any time.}
	\item[number\_sleeping]{Number of spawned threads that are currently sleeping.}
	\item[all\_done]{Boolean-like variable indicating if the entire origin directory has been searched.}
	\item[spawned]{Boolean-like variable indicating if other threads have been spawned yet.}
	\item[threads]{Pointer to array of type \ttt{pthread}, containing one element for each thread spawned.}
	\item[wake\_up]{Signal to wake threads up from sleeping.}
	\item[num\_sleeping]{Mutex protecting \ttt{number\_sleeping}.}
	\item[sleepers]{Pointer to array of mutexes protecting \ttt{number\_sleeping}.}
	\item[done]{Mutex protecting \ttt{all\_done}.}
\end{description}
}
\end{center}

In order to provide each thread with the appropriate information, a custom
datatype was created as shown below.
%
\hlst{12}{25}
%
Similar to previous assignments, the following pieces of information were
passed to each thread:
%
\begin{center}
\parbox{.85\textwidth}{
\begin{description}
	\item[threadID]{Integer identifier for the thread.  The first thread
	spawned will receive ID~0, and subsequent threads will receive~1,~2,
	$\ldots$ , \ttt{NUM\_THREADS} - 1.}
	\item[queue]{Pointer to the queue of files and directories to search.}
	\item[mutex\_queue]{Pointer to the mutex protecting the work queue.}
	\item[num\_occurrences]{Pointer to the integer representing the number of occurrences of the target string.}
	\item[mutex\_count]{Pointer to the mutex protecting~\ttt{num\_occurrences}.}
\end{description}
}
\end{center}
%
Additionally, a custom datatype was created to ease the process of generating
scriptable output.
%
\hlst{27}{34}
%
While these instance variables are fairly self-explanatory, they were created
with the following intended meanings:
%
\begin{center}
\parbox{.85\textwidth}{
\begin{description}
	\item[count]{The number of occurrences of the target string found by the returning function.}
	\item[time]{Amount of time required to execute the returning function.}
\end{description}
}
\end{center}

Finally, the non-main functions were declared:
%
\hlst{36}{39}
%
in which~\ttt{search\_for\_string\_serial} is the single-threaded
implementation,~\ttt{search\_for\_string\_mt} is the multi-threaded
implementation, and~\ttt{worker\_thread} is the function executed by each
thread.

\subsection{Main File}
\subsubsection{Main}
The main entry point for the search test is, as is expected, the~\ttt{main}
function, shown below.
%
\clst{24}{46}
%
The function first checks the command line arguments to ensure that it has
sufficient information to run~--- a string and a target directory.  If this
information is not provided, the usage syntax is printed to the screen and the
application exists.  Additionally, if the number of threads is not provided, it
defaults to eight.  Once this initial validation is complete, the single- and
multi-threaded implementations are executed, and a script-safe output in the
format of the following example:
%
\begin{table}[H]
	\centering
	\begin{tabular}{ccccc}
		                    & Threads & Single & Multi & Match \\
		Scriptable output:  & 8   & 3.292550   & 3.163284 & true \\
	\end{tabular}
\end{table}
%
This provides the number of threads, execution time for both the single- and
multi-threaded implementations, and whether or not the number of occurrences
for the two implementations matches in an easily-parsable format.  While this
was not a requirement for the assignment, it made the process of acquiring and
analyzing the resulting data much easier.

\subsubsection{Single-Threaded}
There were no appreciable changes made to the single-threaded implementation.
While the return type was changed to~\ttt{RET\_TYPE} and there were slight
modifications to the post-search code, the searching portions of the function
remained unmodified from the provided version.

\subsubsection{Multi-Threaded}
The multi-threaded approach to string search begins by allocating space for the
threads array and declaring and initializing the relevant mutexes.
%
\clst{194}{226}
%
Next, an array of mutexes for sleeping threads is created, as well as a series
of counters.  Finally, an instance of~\ttt{THREAD\_ARGS} is created and filled
with the appropriate values.

At this point in the execution, a timer is created and started, and an initial
thread is spawned.
%
\clst{228}{237}
%
After spawning the first thread, the multi-threaded control function simply
waits for all threads to return, at which point it stops the timer, calculates
the execution time, and prints its final search results and execution time.
%
\clst{239}{259}

\subsubsection{Worker Thread}
\paragraph{Setup}
Each thread begins simply by casting the arguments of~\ttt{worker\_thread} from
void~(as required by the pthread libraries) back to type~\ttt{THREAD\_ARGS}.
%
\chompclst{1}{9}
%
As in the single-threaded case, two queue elements are first declared as well
as the relevant objects for any directory information that will be encountered.

If the calling thread is the first one spawned~(i.e. with a \ttt{threadID} of
zero), the shared work queue is allocated and initialized.
%
\chompclst{12}{27}
%
While this could have also been performed before any threads were spawned, it
was included here in the interest of clarity.  Testing for the thread's ID
happens once per thread and requires only~$O(1)$ time, so this causes no
significant slowdown.

\paragraph{The Loop}
While the implementation of the searching algorithm required modifications in
order to prevent deadlocking and data corruption, its overall flow nd operation
remained largely unchanged.  For this reason, only the changes will be
discussed here.

Each thread begins by removing the first item of the queue.  Because all threads access
and modify the same queue, the mutex referenced by~\ttt{mutex\_queue} must be
locked prior to access and unlocked after in order to prevent conflicts.
%
\chompclst{30}{37}
%
While the single-threaded implementation tested the state of the queue in
the~\ttt{while} loop's condition, the test was moved to the first operation
inside the loop so that the queue could be properly protected by its mutexes.

The next change occurs when the end of a directory is reached, in which the
initial thread spawns its siblings.
%
\chompclst{40}{66}
%
Each thread tests its ID and whether or not the extra threads have been
spawned.  If it is thread zero and the sibling threads haven't been created~(as
indicated by the value of the~\ttt{spawned} variable), then an instance
of~\ttt{THREAD\_ARGS} is created and filled prior to each thread's creation.

For each new element reached by a thread, the queue must be locked and
subsequently unlocked in order to contribute to the work queue.
%
\chompclst{69}{77}
%
Once the newly constructed queue element has been added and the queue unlocked,
all remaining threads are woken up with an iterated call
to~\ttt{pthread\_cond\_signal}.  While~\ttt{pthread\_cond\_broadcast} should be
used to wake multiple threads that are waiting on one signal, it does not work
in this case~--- this is most likely caused by the fact that each thread locks
a different mutex prior to calling~\ttt{pthread\_cond\_wait}.

In the event that the queue is empty, a thread simply unlocks the queue's mutex
and locks the mutexes protecting the number of sleeping threads and the
completion status.
%
\chompclst{85}{96}
%
If the process is complete, the two mutexes are unlocked~(in the reverse order
of locking to prevent deadlocks) and the thread exits.  If the process is not
complete but the currently-executing thread is the only remaining thread awake,
it sets the completion status to true before unlocking the two previously
locked mutexes.
%
\chompclst{99}{110}
%
As it is the only thread not sleeping, the executing thread serially wakes the
remaining threads before exiting.

For the case that a thread is not the last one awake and the search process is
not complete, it~(safely) increments the number of sleeping threads.  It next
locks the mutex protecting its sleeping state and safely checks the completion
state to ensure that the search hasn't completed since its last check.  This
prevents a thread from entering an eternal sleep.
%
\chompclst{113}{141}
%
If the search still has not completed then the thread waits for
a~\ttt{wake\_up} signal, essentially putting it to sleep until another thread
wakes it.  Upon waking, a thread will either exit if the search is complete or
decrement the number of sleeping threads and continue from the beginning of the
loop once again.

\section{Results}
Initial testing of the single- and multi-threaded implementations was tested on
my local development machine, but the final data was acquired through the
College of Engineering's Linux cluster.  A recursive search beginning
at~\ttt{/home/DREXEL/nk78} for the string~``Kandasamy'' was performed one
hundred times for maximum thread counts of one, two, four, and eight.  By
acquiring the data at roughly~2:00~PM on Thanksgiving Day, the number of
processes contending for the cluster's CPU's was minimized.  The resulting
average, minimum, and maximum execution times are plotted in
Figure~\ref{f:plot} and tabulated in Table~\ref{t:data}.
%
\begin{figure}[H]
\centering
	\begin{tikzpicture}[gnuplot]
%% generated with GNUPLOT 4.4p2 (Lua 5.1.4; terminal rev. 97, script rev. 96a)
%% Thu 01 Dec 2011 09:02:31 PM EST
\gpsolidlines
\gpcolor{gp lt color border}
\gpsetlinetype{gp lt border}
\gpsetlinewidth{1.00}
\draw[gp path] (1.320,0.985)--(1.500,0.985);
\node[gp node right] at (1.136,0.985) { 2};
\draw[gp path] (1.320,1.461)--(1.500,1.461);
\node[gp node right] at (1.136,1.461) { 4};
\draw[gp path] (1.320,1.936)--(1.500,1.936);
\node[gp node right] at (1.136,1.936) { 6};
\draw[gp path] (1.320,2.412)--(1.500,2.412);
\node[gp node right] at (1.136,2.412) { 8};
\draw[gp path] (1.320,2.888)--(1.500,2.888);
\node[gp node right] at (1.136,2.888) { 10};
\draw[gp path] (1.320,3.363)--(1.500,3.363);
\node[gp node right] at (1.136,3.363) { 12};
\draw[gp path] (1.320,3.839)--(1.500,3.839);
\node[gp node right] at (1.136,3.839) { 14};
\draw[gp path] (1.320,4.314)--(1.500,4.314);
\node[gp node right] at (1.136,4.314) { 16};
\draw[gp path] (1.320,4.790)--(1.500,4.790);
\node[gp node right] at (1.136,4.790) { 18};
\draw[gp path] (1.320,0.985)--(1.320,1.165);
\node[gp node center] at (1.320,0.677) { 0};
\draw[gp path] (2.311,0.985)--(2.311,1.165);
\node[gp node center] at (2.311,0.677) { 2};
\draw[gp path] (3.303,0.985)--(3.303,1.165);
\node[gp node center] at (3.303,0.677) { 4};
\draw[gp path] (4.294,0.985)--(4.294,1.165);
\node[gp node center] at (4.294,0.677) { 6};
\draw[gp path] (5.285,0.985)--(5.285,1.165);
\node[gp node center] at (5.285,0.677) { 8};
\draw[gp path] (6.277,0.985)--(6.277,1.165);
\node[gp node center] at (6.277,0.677) { 10};
\draw[gp path] (7.268,0.985)--(7.268,1.165);
\node[gp node center] at (7.268,0.677) { 12};
\draw[gp path] (8.259,0.985)--(8.259,1.165);
\node[gp node center] at (8.259,0.677) { 14};
\draw[gp path] (9.251,0.985)--(9.251,1.165);
\node[gp node center] at (9.251,0.677) { 16};
\draw[gp path] (10.242,0.985)--(10.242,1.165);
\node[gp node center] at (10.242,0.677) { 18};
\draw[gp path] (1.320,4.790)--(1.320,0.985)--(10.242,0.985)--(10.242,4.790)--cycle;
\node[gp node center,rotate=-270] at (0.246,2.887) {Time (s)};
\node[gp node center] at (5.781,0.215) {Number of Threads};
\node[gp node center] at (5.781,5.252) {Multithreaded Recursive Text Search};
\gpcolor{gp lt color 0}
\gpsetlinetype{gp lt plot 0}
\draw[gp path] (1.816,3.374)--(1.816,4.339);
\draw[gp path] (1.726,3.374)--(1.906,3.374);
\draw[gp path] (1.726,4.339)--(1.906,4.339);
\draw[gp path] (2.311,2.209)--(2.311,2.567);
\draw[gp path] (2.221,2.209)--(2.401,2.209);
\draw[gp path] (2.221,2.567)--(2.401,2.567);
\draw[gp path] (3.303,1.369)--(3.303,1.730);
\draw[gp path] (3.213,1.369)--(3.393,1.369);
\draw[gp path] (3.213,1.730)--(3.393,1.730);
\draw[gp path] (5.285,1.213)--(5.285,1.596);
\draw[gp path] (5.195,1.213)--(5.375,1.213);
\draw[gp path] (5.195,1.596)--(5.375,1.596);
\draw[gp path] (9.251,1.403)--(9.251,1.633);
\draw[gp path] (9.161,1.403)--(9.341,1.403);
\draw[gp path] (9.161,1.633)--(9.341,1.633);
\gpsetpointsize{4.00}
\gppoint{gp mark 1}{(1.816,3.623)}
\gppoint{gp mark 1}{(2.311,2.398)}
\gppoint{gp mark 1}{(3.303,1.564)}
\gppoint{gp mark 1}{(5.285,1.260)}
\gppoint{gp mark 1}{(9.251,1.473)}
\gpcolor{gp lt color border}
\gpsetlinetype{gp lt border}
\draw[gp path] (1.320,4.790)--(1.320,0.985)--(10.242,0.985)--(10.242,4.790)--cycle;
%% coordinates of the plot area
\gpdefrectangularnode{gp plot 1}{\pgfpoint{1.320cm}{0.985cm}}{\pgfpoint{10.242cm}{4.790cm}}
\end{tikzpicture}
%% gnuplot variables

	\parbox{.6\textwidth}{
	\caption{Mean, minimum, and maximum execution times from a set of~100 tests
	executed on~\ttt{xunil-01}.}
	\label{f:plot}}
\end{figure}
%
As is shown, there is a nearly exponential decay in mean execution time over
the range from one to eight threads, causing the average value to decrease
from~\SI{13.09}{\second} to~\SI{3.15}{\second}.  This falls very much in line
with one's expectations about parallelizing an algorithm such as the one used
here.  Single threaded, it takes roughly~$O(n)$ time; by adding~$i$ additional
threads, this drops to~$O(\frac{n}{i+1})$, as each thread can
contribute~$\frac{1}{i+1}$ of the work in a well-balanced program.
%
\begin{table}[H]
\centering
	\begin{tabular}{|c|c|c|c|}
	\\hline
	\tbf{Number of Threads} &
	\tbf{Average (s)} & \tbf{Minimum (s)} & \tbf{Maximum (s)} \\ \hline
	1 		& 	13.09095947 	& 	12.047517 	& 16.104431 \\ \hline
	2 		& 	7.94123965 		& 	7.145596 	& 8.650321  \\ \hline
	4 		& 	4.43541181 		& 	3.615055 	& 5.133428  \\ \hline
	8 		& 	3.1560362 		& 	2.960557 	& 4.570848  \\ \hline
	16 		& 	4.05127182 		& 	3.758859 	& 4.725392  \\ \hline
\end{tabular}

	\parbox{.6\textwidth}{
	\caption{Data used to generate the plot in Figure~\ref{f:plot}.}
	\label{t:data}}
\end{table}
%
There is however a discontinuity to this nearly exponential decay, located
where the search was performed with~16 threads.  Intuition states that this
test case would perform~\emph{better} than any of the cases with fewer threads,
but this is clearly not the case.  Such a slow down is most likely caused by
the added contention for mutexes introduced by multi-threading the process.
All of the threads share a mutex protecting the completion status, one
protecting the number of sleeping threads, and most importantly, a single work
queue.  As a result, only one thread can read from or post to the work queue at
a time (similarly with checking or setting the completion status or the number
of sleeping threads).  As the number of threads increases, the contention for
these limited resources becomes a bottleneck on the overall performance,
eventually resulting in threads that spend more time waiting for shared
resources than actually doing work.

\newpage
\appendix

\section{Scripts}
Data for this report was acquired and manipulated through a series of Bash and
Python scripts, which have been included here for transparency.

\subsection{Data Acquisition}
\begin{center}
	\parbox{.85\textwidth}{
	\lstinputlisting[language=bash]{../data/getData.sh}}
\end{center}

\subsection{Data Processing}
\begin{center}
	\parbox{.85\textwidth}{
	\lstinputlisting[language=bash]{../data/processData.sh}}
\end{center}

\subsection{Data Calculation}
\begin{center}
	\parbox{.85\textwidth}{
	\lstinputlisting[language=python]{../data/calcData.py}}
\end{center}

\newpage
\section{License}
Copyright \copyright\ 2011, Sean Barag.  All rights reserved.

Redistribution and use in source and binary forms, with or without
modification, are permitted provided that the following conditions are met:
\begin{itemize}
\item Redistributions of source code must retain the above copyright notice, this
  list of conditions and the following disclaimer.
\item Redistributions in binary form must reproduce the above copyright notice, this
  list of conditions and the following disclaimer in the documentation and/or
  other materials provided with the distribution.
\item Neither the name of the owner nor the names of its contributors may be
  used to endorse or promote products derived from this software without specific
  prior written permission.
\end{itemize}

THIS SOFTWARE IS PROVIDED BY THE COPYRIGHT HOLDERS AND CONTRIBUTORS ``AS IS'' AND
ANY EXPRESS OR IMPLIED WARRANTIES, INCLUDING, BUT NOT LIMITED TO, THE IMPLIED
WARRANTIES OF MERCHANTABILITY AND FITNESS FOR A PARTICULAR PURPOSE ARE
DISCLAIMED. IN NO EVENT SHALL THE COPYRIGHT HOLDER OR CONTRIBUTORS BE LIABLE
FOR ANY DIRECT, INDIRECT, INCIDENTAL, SPECIAL, EXEMPLARY, OR CONSEQUENTIAL
DAMAGES (INCLUDING, BUT NOT LIMITED TO, PROCUREMENT OF SUBSTITUTE GOODS OR
SERVICES; LOSS OF USE, DATA, OR PROFITS; OR BUSINESS INTERRUPTION) HOWEVER
CAUSED AND ON ANY THEORY OF LIABILITY, WHETHER IN CONTRACT, STRICT LIABILITY,
OR TORT (INCLUDING NEGLIGENCE OR OTHERWISE) ARISING IN ANY WAY OUT OF THE USE
OF THIS SOFTWARE, EVEN IF ADVISED OF THE POSSIBILITY OF SUCH DAMAGE.\\

Source code for this document is available at \texttt{http://github.com/sjbarag/}.


\end{document}

